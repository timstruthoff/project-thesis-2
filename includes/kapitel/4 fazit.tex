\chapter{Fazit}

In dieser Arbeit konnte erfolgreich eine REST API untersucht und verwendet werden. Es wurden zunächst die theoretischen Grundlagen und Ursprünge des REST Architekturstils untersucht. Mit diesen Grundlagen konnten dann Regeln zur Umsetzung von REST Schnittstellen aus Literatur herausgearbeitet und begründet werden.

Diese Regeln wurden dann auf die Aruba Central REST Schnittstelle angewendet. Es konnte bewiesen werden, dass die Schnittstelle mit den allgemeinen Regeln übereinstimmt.

Da die Schnittstelle REST-konform ist, konnte mit einem JavaScript Programm darauf zugegriffen werden. Es konnte ein Prototyp erstellt werden, der es dem Kunden ermöglicht, alle ihm erforderlichen Statusinformationen über die APs abzurufen.

Das entstandene Programm wurde letztendlich vom Kunden leicht angepasst und in die Inventarisierungssoftware integriert. So kann der Kunde mit dem Arbeitsergebnis automatisiert die Aruba Access Points überwachen. Dieser Prozess würde bei einer manuellen Umsetzung deutlich mehr Arbeitszeit kosten. So konnte mit der Umsetzung und Integration des vorliegenden JavaScript-Programmes ein ökonomischer Nutzen erzielt werden. Nach der erfolgreichen Integration wurde vom Kunde exzellentes Feedback geäußert.

\section{Kritische Evaluierung}

Wird vom Kunden verwendet
Einordnung nur heuristik
Es wurde nur ein Endpunkt analysiert

\section{Ausblick}

Das, in dieser Arbeit vorgestellte Programm stellt lediglich einen Prototypen dar. Dieser wurde bereits vom Kunden integriert und wird nun verwendet und weiter entwickelt. So wurden alle Ziele der Arbeit erfüllt. 

Dennoch könnte das Arbeitsergebnis erweitert werden: Das Skript könnte generalisiert werden und als Beispiel in die öffentliche Dokumentation aufgenommen werden. Das würde es anderen Kunden von Aruba ermöglichen die gleiche Problemstellung zu bewältigen.

Weiter existiert bereits eine Programmbibliothek für die Programmiersprache Python, welche es ermöglicht, häufig benutzte Funktionen der Aruba Central Schnittstelle abstrahiert zu bedienen. Eine ähnliche Programmbibliothek könnte auch für die Programmiersprache JavaScript erstellt werden. Das würde es den Kunden von Aruba ermöglichen auch eine Vielzahl von Problemstellungen mittels der Aruba Central Schnittstelle zu bewältigen.
