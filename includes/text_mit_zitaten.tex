\chapter{Zitieren}\label{chapter:zitate}

Der Zitierstil ist so angepasst, dass er den Zitierrichtlinien des Studiengangs Wirtschaftsinformatik der DHBW Stuttgart entspricht. 

\section{Zitate in den Text einfügen}
In \LaTeX\ wird mit den Befehlen \verb|\footcite| 
oder 
\verb|\cite|
eine Referenz im Text eingefügt. Meist wird \verb|\cite| nur \emph{innerhalb} einer Fußnote benutzt. 
Damit ein vorangestelltes \enquote{Vgl.} in der Fußnote erscheint, können Sie wie folgt zitieren:
\begin{verbatim}
\footcite[Vgl.][S. 3]{Autor}
\footcite[Vgl.][]{Autor}
\end{verbatim}

Das erste optionale Argument von \verb|\footcite| wird dem Zitat vorangestellt, das zweite ist die Seitenzahl. Den selben Effekt hätte
\begin{verbatim}
\footnote{Vgl. \cite[S. 3]{Autor}}
\footnote{Vgl. \cite{Autor}}
\end{verbatim}

Hinweis: Falls \enquote{Vgl.}, aber keine Seitenzahl angeben werden soll, muss das zweite Argument vorhanden (jedoch leer) sein, ansonsten wird \enquote{Vgl.} als Seitenzahl interpretiert. Falsch ist also: 
\begin{verbatim}
\footcite[Vgl.]{Autor}  % so nicht!
\end{verbatim}



\subsection{Beispiele}
Nachfolgend ein paar Beispiele, um die korrekte Darstellung zu überprüfen:

\begin{itemize}
\item \cite{chaniotis_is_2015} ist ein Buch über \LaTeX.
\item Zur Vorlesung \emph{Logik und Algebra} gibt es das gleichnamige Lehrbuch.\footcite{Staab}
\end{itemize}

\subsection{Spezialfälle}
\begin{itemize}
\item \emph{Zwei Quellen am Satzende} werden durch Komma getrennt.\footcite{Staab}${}^{,}$\footcite{mayerLukas:PA1} Hier muss \verb|${}^{,}$| eingeschoben werden.
\item \emph{Eindeutigkeit:} Normalerweise wird kein Vorname des Autors angegeben. Falls es allerdings zur Eindeutigkeit\footnote{\cite{trautwein2011unternehmensplanspiele} vs. \cite{hitzler2011optimierung}} (bei gleicher Jahreszahl) erforderlich ist, wird der Vorname abgekürzt bzw.\ nötigenfalls sogar ganz ausgeschrieben mit angegeben.\footnote{Vgl. \cite{mayer:PA1} und \cite{mayerLukas:PA1}}
 
Welch ein Glück, dass Sie sich darum dank \LaTeX\ gar nicht kümmern müssen (arme Word\texttrademark-User ;-).

\item Die Verwendung von \emph{Sekundärliteratur}\footcitePrimaerSekundaer{Primaerquelle}{}{Sekundaerquelle}{11} wird weiter inhttps://www.overleaf.com/project/609e84b7f494b816c7da944f Abschnitt~\ref{sec:sekundaerliteratur} erläutert.
\end{itemize}


\section{Eintragstypen für die Literatur-Datenbank}

Die verwendete Literatur pflegen Sie in einer Literatur-Datenbank im Bibtex-Format. Dabei handelt es sich um eine Textdatei, wobei für jede Quelle mittels Name-Value-Pairs die relevanten Attribute (Autor, Titel etc.) hinterlegt sind. Die Datei wird üblicherweise nicht im Texteditor, sondern in einem spezialisierten Programm wie JabRef bearbeitet.

Sofern in der Literatur-Datenbank der Typ eines Eintrags (Entry Type) korrekt festgelegt ist, wird er im Literaturverzeichnis automatisch richtig dargestellt. Mit folgenden Typen sollten Sie i.d.R.\ auskommen:
\begin{description}
\item[article] Artikel in einer Fachzeitschrift, auch E-Journal (Zeitschrift in elektronischer Form)\footnote{Bei E-Journals/E-Books werden beim Zitieren anstelle der (u.U. nicht eindeutigen, da von der Schriftgröße abhängigen) Seitenzahl Abschnitt und Absatz näher bezeichnet, also: \cite[Abschnitt 1.2.3, Absatz 4]{Staab}.}
\item[book] Buch, auch E-Book 
\item[inbook] Kapitel in einem Buch, zu dem mehrere Autoren beigetragen haben 
\item[inproceedings] Beitrag zu einer Fachtagung/Konferenz 
\item[manual] Handbuch
\item[misc] anderweitig nicht zuordenbarer Typ
\item[phdthesis] Dissertation
\item[thesis] Bachelor-/Master-/Diplomarbeit (Art wird im Attribut \enquote{type} festgelegt) 
\item[online] Internet- oder Intranet-Quelle\footnote{\label{fn:onlineEntryType}Man beachte, dass der Eintragstyp \enquote{online} in JabRef nur im \enquote{biblatex-Modus} (Menü: Datei -- Neue biblatex Bibliothek) auswählbar ist.}
\item[report] technischer Bericht, Forschungsbericht oder White Paper; diesen Typ können Sie auch verwenden, um eine Projektarbeit zu zitieren (Art wird im Attribut \enquote{type} festgelegt) 
\end{description} 

Eine Übersicht über die notwendigen Attribute jedes Eintragstyps gibt die folgende Tabelle, wobei ein Schrägstrich als \enquote{oder} zu verstehen ist.\footnote{Auszugsweise entnommen aus \cite{biblatex:manual}.} Zudem sind die wichtigsten optionalen Attribute aufgeführt.

\begin{tabular}{l|p{6cm}|p{6cm}}
\textbf{Eintragstyp} & \textbf{notwendige Attribute} & \textbf{optionale Attribute (Auswahl)} \\
\hline
article & author, title, journal, year/date  & volume, number, pages, month, note \\
\hline
book & author, title, year/date  & publisher, edition, editor, \mbox{volume/number}, series, isbn, url \\
\hline
inbook & author, title, booktitle, year/date  & bookauthor, editor, volume/number, series, isbn, url \\
\hline
inproceedings & author, title, booktitle, year/date  & organization/publisher, editor, volume/number, series, isbn, url  \\
\hline
manual & author/editor, title, year/date  & organization/publisher, address, edition, month, note, url, urldate \\
\hline
misc & author/editor, title, year/date & howpublished, organization, month, note \\
\hline
phdthesis & author, title, institution, year/date  & address, month, note \\
\hline
thesis & author, title, institution, type, year/date &  address, month, note \\
\hline
online\textsuperscript{\ref{fn:onlineEntryType}} & author/editor, title, year\footnotemark/date, url & urldate \\
\hline
report & author, title, institution, type, year/date & number, version, url, urldate \\
\end{tabular}
\footnotetext{%
Sofern kein Jahr bekannt ist, sollte das Attribut nicht leer gelassen werden (sonst wird die aktuelle Jahreszahl automatisch eingefügt), sondern der Eintrag \enquote{o.J.} gewählt werden.}



\section{Zitieren von Sekundärliteratur}\label{sec:sekundaerliteratur}
Gelegentlich lässt es sich nicht vermeiden, aus der Sekundärliteratur zu zitieren. Dies leistet der folgende Befehl.
\begin{verbatim}
\footcitePrimaerSekundaer{Primaerquelle}{Seite}{Sekundaerquelle}{Seite}
\end{verbatim}
Die erste Seitenangabe bezieht sich auf die Primär-, die zweite auf die Sekundärquelle.
Die Seitenangaben sind optional, sie können auch leer bleiben.\footcitePrimaerSekundaer{Primaerquelle}{23}{Sekundaerquelle}{}
Es ist aber zu beachten, dass der Befehl \verb|\footcitePrimaerSekundaer| vier Argumente hat.

Ins Literaturverzeichnis soll nur die Sekundärquelle aufgenommen werden. Dies wird dadurch erreicht, dass in der Literatur-Datenbank bei der Primärquelle im Attribut \enquote{keyword} der Wert \enquote{ausblenden} eintragen wird.
